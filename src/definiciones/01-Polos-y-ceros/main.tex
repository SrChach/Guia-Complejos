\subsubsection{Ceros}

Un punto $Z_0$ en el plano complejo es un cero de orden $n$ \textit{si}:

\begin{align}
    f(Z_0) &= f'(Z_0) = f''(Z_0) = ... = f^{n-1}(Z_0) = 0 \label{eq:ceros-1} \\
    f^{n}(Z_0)& \ne 0 \label{eq:ceros-2}
\end{align}

La ecuación y todas sus derivadas hasta la derivada $n-1$ son cero (ec \ref{eq:ceros-1}), pero la \textit{enésima} derivada es diferente de cero (ec \ref{eq:ceros-2}).

\subsubsection{Polos}

Se llama \textit{"polos"} o \textit{"singularidades aisladas"} a los puntos ($"Z_0"$ en este caso) en los cuales, al evaluar la función, el denominador se vuelve $0$. Esto hace que la función no pueda ser evaluada en ese punto, y tienda hacia infinito de la siguiente manera

\begin{align}
    f(Z) &= \frac{g(Z)}{(Z-Z_0)^n} = \frac{g(Z)}{0} \label{eq:polos-1}
\end{align}

\textbf{Donde:}

\begin{addmargin}[2em]{2em}
    $\pmb{Z_0}$ es el punto donde el denominador se vuelve cero ($"Z - Z_0 = 0"$) y se indetermina la ecuación \\
    $\pmb{n}$ es un real entero, y se llama el \textit{"orden"} del polo \\ \\
    Condicion: $\pmb{g(Z_0)}$ NO puede ser cero en el punto $Z_0$
\end{addmargin}

% Pendientes:
%   Multiples polos en la función
%   Ejemplos

% Pendientes externas
%   Desarrollo en series geométricas de Sin (Para el ejercicio Sin(z)/z se desarrolla en serie el numerador, como z())