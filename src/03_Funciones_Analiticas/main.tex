\section{Funciones Analíticas}

\subsection{Formula de De Moivre}
%%novalidate

\begin{align}
    (Cos\theta + iSen\theta)^n &= Cos(n\theta) + iSen(n\theta)
        \label{eq:de_moivre} \\
    (e^{i\theta})^n &= e^{in\theta}
        \label{eq:de_moivre_e_form} \\
    & \textbf{Donde} \nonumber \\
    \pmb{n}&= \text{un número real entero} \nonumber \\
    \pmb{(\ref{eq:de_moivre})}&= \text{La fórmula, en forma de senos y cosenos} \nonumber \\
    \pmb{(\ref{eq:de_moivre_e_form})}&= \text{La misma Fórmula, en forma euleriana} \nonumber
\end{align}

\subsection{Raíces enésimas}
%%novalidate

%%novalidate

\begin{align}
    W_k &= r^{1/n}[
        Cos(\frac{\theta + 2\pi k}{n}) +
        iSen(\frac{\theta + 2\pi k}{n})
    ] & \label{eq:nth_roots}
\end{align}

\begin{align}
    & \textbf{Donde} \nonumber \\
    \pmb{n}&= \text{el número de raíces a sacar} \nonumber \\
    \pmb{r}&= |z| \text{, el modulo de Z} \nonumber \\
    \pmb{\theta}&= Arg(Z) \mid Z \in [-\pi, \pi] \text{ Argumento principal de Z} \nonumber \\
    \pmb{k}&= (0, 1, 2, ..., n-1) \nonumber \\
    \pmb{W}&= \text{La raíz específica que calcularemos} \nonumber
\end{align}

Para poder aplicar la fórmula, el número debe ser de la forma:
\begin{align}
    W^n = Z \longrightarrow W = Z^{1/n} \nonumber
\end{align}

\subsection{Condiciones de Cauchy-Riemann}
Teorema: Si una función \( f(z) = u(x, y) + iv(x,y) \) cumple con las condiciones (\textbf{\ref{eq:cauchy-riemann-cond-1}}) y (\textbf{\ref{eq:cauchy-riemann-cond-2}}) en un punto \( z_0 = x_0 + iy_0 \), las condiciones de Cauchy-Riemann se cumplen para ese punto

%%novalidate

\begin{align}
    \frac{\partial u}{\partial x} &= \frac{\partial v}{\partial y}
        \label{eq:cauchy-riemann-cond-1} \\
    \frac{\partial u}{\partial y} &= -\frac{\partial v}{\partial x}
        \label{eq:cauchy-riemann-cond-2}
\end{align}


Donde (\textbf{\ref{eq:cauchy-riemann-cond-1}}) y (\textbf{\ref{eq:cauchy-riemann-cond-2}}) deben cumplirse, para demostrar que la función es analítica

\subsection{Funciones analíticas}
%%novalidate

Una función $f(z)$ es analítica en una región $R$ si es diferenciable en un 'vecindario' cercano a cada punto en $R$. \\

Si una función es \textit{analítica} en una región $R$, se cumple lo siguiente dentro de $R$

\begin{itemize}
  \item $f'(z)$, $f''(z)$, $...$ las derivadas de cualquier orden existen
  \item $f(z)$ puede ser representada como una serie de potencias
\end{itemize}

\textbf{NOTA}: Se dice que una función es \textit{"entera"} si es analítica en TODO el plano complejo

\subsection{Series de Taylor}

\subsubsection{Definicion}
    Si una función es analítica en un punto $Z_0$, entonces la serie

\begin{align}
    f(Z_0) + \frac{f'(Z_0)}{1!}(Z-Z_0) + \frac{f''(Z_0)}{2!}(Z-Z_0)^2 +
        \frac{f^(3)(Z_0)}{3!}(Z-Z_0)^3 + ...
        = \sum_{n=0}^{\infty} \frac{f^{(n)}(Z_0)}{n!}(Z-Z_0)^n
\end{align}

es llamada la \textit{Serie de Taylor} para f(Z) desarrollada alrededor del punto $Z_0$

\subsubsection{Teorema}
    %%novalidate

Si una función $f(Z)$ es analítica en \( |Z - Z_0| < R \) (donde "$|Z - Z_0| < R$" son todos los puntos dentro del círculo con centro en $Z_0$ y radio $R$), entonces

\begin{align}
    f(Z) = \sum_{n=0}^{\infty} \frac{f^{(n)}(Z_0)}{n!}(Z-Z_0)^n \label{eq:taylor-serie}
\end{align}

$f(Z)$ puede ser representada como una \textit{Serie de Taylor} (\ref{eq:taylor-serie}) para todos los puntos $Z$ en $|Z - Z_0| = R$

\textbf{NOTA:} Ejercicios de este tema en la \hyperref[ex:taylor-series]{página \pageref{ex:taylor-series}}

