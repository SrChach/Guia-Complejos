\subsection{Notación}

Los números complejos son un conjunto de números representados por la letra $\mathbb{C}$, y tienen la forma $\pmb{z = u + iv}$, donde:

\begin{addmargin}[2em]{2em} % 2em left, 2em right
    $\pmb{i}$ es igual a $\sqrt{-1}$ y representa la unidad en el plano imaginario.

    $\pmb{u}$ es la parte real de $\pmb{z}$, representada por $\Re(z)$ o $\operatorname{Re}(z)$
    
    $\pmb{v}$ es la parte imaginaria de $\pmb{z}$, representada por $\Im(z)$ o $\operatorname{Im}(z)$
    
    Tanto $\pmb{"u"}$ como $\pmb{"v"}$ son números reales. Lo que vuelve imaginaria a la parte imaginaria, es la $\pmb{"i"}$ multiplicándolo.
\end{addmargin}

Cabe recalcar que los números reales ($\mathbb{R}$) son un \textit{Subconjunto} de los números complejos.

