\subsection{Explicando \textit{i} más a fondo}

Dado "$\pmb{i} = \sqrt{-1}$", toca dejar claras ciertas propiedades que se aplican a él.

(\textbf{Nota}: Para calcular un numero estoy usando el resultado de la operación anterior)

    \begin{center}
    \begin{tabular}{ ||l||l||l|| } 
        \hline
        $\pmb{i} = \sqrt{-1}$
            & $\pmb{i}^5 = \pmb{i} * \pmb{i}^4 = \sqrt{-1} * 1 = \sqrt{-1}$
                & $\pmb{i}^9 = \sqrt{-1}$ \\
                
        $\pmb{i}^2 = \pmb{i} * \pmb{i} = (\sqrt{-1})(\sqrt{-1}) = -1$
            & $\pmb{i}^6 = \pmb{i} * \pmb{i}^5 = (\sqrt{-1})(\sqrt{-1}) = -1$
                & $\pmb{i}^{10} = -1$ \\
                
        $\pmb{i}^3 = \pmb{i} * \pmb{i}^2 =  (\sqrt{-1})(-1) = -\sqrt{-1}$
            & $\pmb{i}^7 = \pmb{i} * \pmb{i}^6 =  (\sqrt{-1})(-1) = -\sqrt{-1}$
                & $\pmb{i}^{11} = -\sqrt{-1}$ \\
                
        $\pmb{i}^4 = \pmb{i} * \pmb{i}^3 =  (\sqrt{-1})(-\sqrt{-1}) = -(-1) = 1$
            & $\pmb{i}^8 = \pmb{i} * \pmb{i}^7 =  (\sqrt{-1})(-\sqrt{-1}) = -(-1) = 1$
                & $\pmb{i}^{12} = 1$ \\
        \hline
    \end{tabular}
\end{center}

Como se muestra en la tabla, cada $4$ potencias, el valor de $\pmb{i}$ se repite. De forma que se seguirán ciclando así continuamente hasta el infinito.