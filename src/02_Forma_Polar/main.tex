\section{Forma Polar}

\subsection{Notación base}

Los componentes $\pmb{"x"}$ y $\pmb{"y"}$ de un número complejo $z$ pueden también representarse como:

% \label{name}  to label an equation
% \ref{name}    to reference it layer
% \nonumber     no number for that line

\begin{align}
    x = rCos(\theta) \label{eq:polar_not_x} \\
    y = rSen(\theta) \label{eq:polar_not_y}
\end{align}

Donde: 

\begin{addmargin}[2em]{2em} % 2em left, 2em right
    $\pmb{r}$ es la distancia del origen al punto dado

    $\pmb{\theta}$ es el el ángulo generado entre el eje de las X y el punto
\end{addmargin}

Donde, con $\pmb{Cos}$ obtenemos el desplazamiento en $\pmb{x}$ y con $\pmb{Sen}$ el desplazamiento en $\pmb{y}$. \\

Ahora bien, substituyendo los valores \textbf{x} y \textbf{y} de \textbf{(\ref{eq:polar_not_x})} y \textbf{(\ref{eq:polar_not_y})} en $"\pmb{z = x + iy}"$ obtenemos

\begin{align}
    z &= rCos\theta + riSin\theta \nonumber\\
    z &= r(Cos\theta + iSin\theta) \label{eq:polar_base}
\end{align}

De esa forma, con la ecuación (\ref{eq:polar_base}) obtenemos la base de la forma polar

% \begin{align}
%     z &= rCos\theta + riSin\theta \nonumber
%         && \textit{
%             substituyendo
%             (\ref{eq:polar_not_x}) y (\ref{eq:polar_not_y})
%         } \\
%     z &= r(Cos\theta + iSin\theta)
% \end{align}

% \begin{equation}
% \left.%
% \begin{array}{@{}r@{\quad}ccrr@{}}
% \textrm{a}) & y & = & c & (constant)\\
% \textrm{b}) & y & = & cx+d & (linear)\\
% \textrm{c}) & y & = & bx^{2}+cx+d & (square)\\
% \textrm{d}) & y & = & ax^{3}+bx^{2}+cx+d & (cubic)
% \end{array}%
% \right\} \textrm{Polynomes}
%  \end{equation}

