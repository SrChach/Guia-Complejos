\subsection{Completando cuadrados}

\textbf{Pasos:}

1.- El coeficiente del término al cuadrado debe ser \textit{1}. Si no lo es, habrá que factorizar para dejarlo en esa forma.

2.- Buscar el coeficiente del término lineal. De esta forma, por ejemplo, en el siguiente polinomio
\begin{align*}
    x^2 \underbrace{- 6x} + 8
\end{align*}

el término lineal es "-6".

Tomamos el término lineal (-6 en este caso), lo dividimos sobre dos y lo elevamos al cuadrado, de esta forma:

\begin{align*}
    -6 \longrightarrow \left( \frac{-6}{2} \right)^2
    \longrightarrow 9
\end{align*}

3.- El número que obtuvimos lo sumamos y restamos a la ecuación, como un "0" mañoso.

\begin{align*}
    (x^2 - 6x \underbrace{+ 9} ) + 8 \underbrace{- 9}
\end{align*}

4.- Obtenemos el cuadrado perfecto de la ecuación entre paréntesis. También simplificamos el resto

\begin{align*}
    (x^2 - 6x + 9) + 8 - 9 \longrightarrow (x - 3)^2 - 1
\end{align*}

Así tenemos un término al cuadrado perfecto, mas un sobrante



