\textbf{Instrucciones:} Determine la \textit{Serie de Taylor} alrededor del punto $"Z_0 = 0"$ de la siguiente función y halle su \textit{Radio de Convergencia} (R)

\[ h(Z) = \frac{Z-2}{(Z+2)(Z+3)} \]

\textbf{NOTA:} En este ejemplo veremos qué podemos hacer si no podemos reescribir la función en forma de \textit{Serie Geométrica} o una \textit{Serie de Potencias} conocida.

\textbf{Paso 1}: Encontrar el radio, y la singularidad más cercana.

La función dada tiene dos singularidades, una en $Z = -2$ y otra en $Z = -3$. Ya que desarrollaremos alrededor del punto $Z=0$, la singularidad más cercana está en $Z = -2$.

Por tanto, $R$ es la distancia de 0 a -2, osea $R = 2$

\textbf{Paso 2}:
Ya que no podemos pasarlo directamente a la forma geométrica, usaremos \textit{Fracciones Parciales}

\begin{align}
    \frac{Z-2}{(Z+2)(Z+3)} &= \frac{A}{Z+2} + \frac{B}{Z+3}
        &\text{ Planteamos las fracciones parciales} \nonumber \\
    \Bigl[
            \frac{Z-2}{(Z+2)(Z+3)} &= \frac{A}{Z+2} + \frac{B}{Z+3}
        \Bigr] (Z+2)(Z+3)
        & \text{Multiplicamos por el denominador original} \nonumber \\
    (Z-2) \cancelto{1}{\frac{(Z+2)}{(Z+2)}} \cancelto{1}{\frac{(Z+3)}{(Z+3)}}
        &=A(Z+3) \cancelto{1}{\frac{Z+2}{Z+2}} 
        + B(Z+2)\cancelto{1}{\frac{Z+3}{Z+3}}
        & \text{Anulamos términos} \nonumber \\
    (Z-2) &= A(Z+3) + B(Z+2)
\end{align}
